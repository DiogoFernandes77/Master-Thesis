

\chapter{Metodologia}
\label{chapter:metodolia}




\section{Sistema de Teste}


Fazer uma pequena introdução.

Para efetuar o treinamento da rede foram usados os dados providenciados pela Renault Cacia.
	Cada caixa de velocidades passa por um teste realizado pela Renault em uma das suas linhas testagem. De forma simplificada, o teste consiste em fazer o engrenamento e a passagem de cada uma das mudanças de forma decrescente começando pela mudança mais alta (6ª) e terminando na marcha-atrás. Quando uma mudança está engrenada, a caixa é sujeita a uma certa rotação e binário e são feitas as medições correspondentes às métricas referidas na Análise Vibratória \ref{Analise Vibratória}. Após isto, é feita a passagem e o engrenamento da próxima mudança. Durante a passagem da mudança, ou seja, a manipulação do comandos por parte do colaborador, são feitas as medições das métricas relativamente à passagem de mudança * falta adicionar esta parte no estado de arte *. Após o teste estar concluído, a caixa pode ter uma das 3 seguintes classificações: 
\begin{itemize}
\item \textbf{caixa  boa}, nenhuma das métricas passou os limites estabelecidos por análise estatística pela Renault e o colaborador validou a caixa, não tendo pressentido nenhuma anomalia.
\item \textbf{caixa com defeito}, uma das métricas passou os limites estabelecidos por análise estatística pela Renault, ou o colaborador pressentiu uma anomalia com a caixa.
\item \textbf{teste não conclusivo}, o teste não terminou ou foi interrompido pelo colaborador devido ao não cumprimento de condições de teste. Exemplo: a caixa por ser nova, apresenta algum resíduo de fabrico (ex. limalha) entre os dentes das engrenagens, no qual irá se desintegrar após a caixa entrar em rodagem algumas vezes. 
\end{itemize}

A informação relativamente ao teste de uma caixa de velocidade está armazenada num ficheiro. A figura \ref{ensaio} demonstra parte de um ensaio, neste caso só contém a informação relativa à 6 velocidade que é a primeira a ser testada como referido anteriormente.


\begin{figure}[H]
\centering
\includegraphics[scale=0.3]{figs/ensaio}
\caption{Parte de um ficheiro correspondente a um ensaio.}\label{ensaio}
\end{figure}









